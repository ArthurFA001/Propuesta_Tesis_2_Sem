\documentclass[12pt,letterpaper]{article}
\usepackage[utf8]{inputenc}

%Documento
\usepackage[spanish,es-tabla,es-lcroman]{babel}
\usepackage[T1]{fontenc}
\usepackage{times}
\usepackage[margin=2.54cm]{geometry}
\usepackage{parskip} %saltos de parrafo
\setlength{\parskip}{\baselineskip}

%Formatos
\usepackage{setspace} %espacios de interlineados
\usepackage[dvipsnames]{xcolor} %paquete para colores
%tablas
\usepackage{array}
\usepackage{multicol}
\usepackage{longtable}

%imagenes
\usepackage{graphicx}
\usepackage{subcaption}
\usepackage{enumerate} %numeracion de tablas y figuras
%Listas
\usepackage[shortlabels]{enumitem}

%hipervinculos
\usepackage{hyperref}%Genera hipervinculos
\hypersetup{colorlinks=true,citecolor=black,filecolor=black,linkcolor=black,urlcolor=blue}

%Referencias
\usepackage{natbib}

%matematicas
\usepackage{amsmath,amssymb,amsfonts}

% - - - - - - - - - - - - - - - - - - - - - - - - - - - - - - - - - - - - - - - - - - - - - - 
%Indice de ecuaciones, figuras y tablas
\numberwithin{equation}{section} 	
\numberwithin{figure}{section}
\numberwithin{table}{section}

\begin{document}
\input{Portada.tex}
\input{TablasContenido.tex}

\newpage
\listoffigures
\addcontentsline{toc}{section}{Índice de figuras}

\newpage
\listoftables
\addcontentsline{toc}{section}{Índice de tablas}

\newpage
\addcontentsline{toc}{section}{Nomenclatura}

% - - - - - - - - - - - - - - - - - - -- - - - - - - - - - - - - - - - - - - - - - - -- - - - -
%Nomenclatura

\textbf{\Large{Nomenclatura}}

\begin{table}[h]

    \centering
    \caption{Siglas y acrónimos}
    \vspace{0.3cm}
    \begin{tabular}{
    >{\centering\arraybackslash}m{3cm}
    >{\centering\arraybackslash}m{12cm}} \hline

\textbf{Siglas} & 
\textbf{Descripción} \\ \hline\hline

   IA & Inteligencia Artificial
\\[0.2cm]

   DL & Aprendizaje Profundo (En inglés Deep Learning)
\\[0.2cm]

   ML & Aprendizaje Automático (En inglés Machine Learning)
\\[0.2cm]

   CNN & Red Neuronal Convolucional (En inglés Convolutional Neuronal Network)
\\[0.2cm]
   RNN & Red Neuronal Recurrente (En inglés Recurrent Neuronal Network)
\\[0.2cm]
   NGS & Secuenciación de nueva generación (En inglés Next Generation Sequencing) 
\\[0.2cm]
   ANN & Red Neuronal Artificial (En inglés Artificial Neuronal Network) 
\\[0.2cm]
TCGA & Programa del Atlas del Genoma del Cáncer (En inglés The Cancer Genome Atlas Program) 
\\[0.2cm]
AE & Auto Codificador (En inglés Auto Encoder) 
\\[0.2cm]
GAN & Red Generativa Antagónica (En inglés Generative Adversarial Network) 
\\[0.2cm]
DNN & Red Neuronal Densa (En inglés Deep Neural Network) 
\\[0.2cm]
MLP & Perceptrón Multicapa (En inglés Multilayer Perceptron) 
\\[0.2cm]



\hline
    \end{tabular}
\end{table}

\newpage

\setcounter{page}{1} %se reinicia el contador de la páginas después de la Tabla de contenido
\pagenumbering{arabic}

% - - - - - - - - - - - - - - - - - - - - - - - - - - - - - - - -
%Documento principal

\input{Introduccion.tex}
\section{Objetivos}

\subsection{Objetivo general}
\begin{itemize}

   \addtolength{\itemsep}{-4mm} %con esto se ajusta el interlineado entre la lista
        \item Modelar datos genómicos, transcriptómicos y proteómicos utilizando técnicas de aprendizaje profundo, empleando redes neuronales de diseño propio, y configurar la salida del modelo para propósitos de clasificación y predicción, enfocados en el diagnóstico y pronostico del cáncer.

    \end{itemize}


\subsection{Objetivos específicos}

\begin{itemize}

   \addtolength{\itemsep}{-4mm} %con esto se ajusta el interlineado entre la lista
        \item Codificar datos genómicos, transcriptómicos y proteómicos para ser alimentados en las redes neuronales.
        \item Implementar redes neuronales convolucionada (CNN) y recurrente (RNN), y entrenarlas.
        \item Proponer y diseñar una red neuronal propia a partir de las dos anteriores.
        \item Configurar en cada caso la salida, para propósito de clasificación y predicción.
        \item Enfocar el modelado de las redes neuronales para fines biomédicos en diagnóstico y pronostico del cáncer.
        \item Validar los resultados con las bases de datos e incluyendo opinión de especialistas.
    \end{itemize}
\section{Marco Teorico}

\subsection{Redes Neuronales Graficas}

Las Redes Neuronales de grafos (GNN) son modelos de aprendizaje profundo en los cuales se procesan datos en representaciones de grafos o redes. Las GNN permiten aprovechar las relaciones que se codifican en los grafos para extraer características significativas y realizar tareas como la clasificación de nodos, la predicción de aristas (conexiones) y la inferencia a nivel de grafo. Este tipo de modelos aprende de las incrustaciones que capturen la informacion estructural del grafo como las características que se tengan en los nodos como se puede observar en la figura \ref{fig:GNN}

\begin{figure}[h!]
    \centering
    \includegraphics[width=0.9\textwidth]{Imagenes/GNN.png}
    \caption{Ilustración de grafo de entrada (a) y el proceso de la GNN en el cual se realiza el cálculo de la representación vectorial del nodo E agregando información de los nodos vecinos (b).}
    \label{fig:GNN}
\end{figure}

\subsubsection{Grafos}
\subsubsection{GAT}
\subsubsection{GCN}
\subsubsection{GTN}
\section{Avances}

\subsection{Procesamiento de datos ómicos}

\subsection{Estructuras DL en datos ómicos}
\input{ActividadesFuturas.tex}
\input{Cronograma.tex}


\newpage

\bibliographystyle{apalike} %eslito del citado %apacite %plain %apalike
\bibliography{bibliografia.bib} %archivo de las referencias

\end{document}