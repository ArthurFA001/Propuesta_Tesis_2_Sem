\section{Propuesta de tesis}

%\subsection{Antecedentes}

%El aprendizaje profundo actualmente tiene como base de procesamiento a las redes neuronales, en CENIDET se cuentan con trabajos relacionados con la aplicación de algoritmos de redes neuronales artificiales.

%Uno de los principales intereses es la participación con especialistas que participan en el área médica en el centro oncológico de San Peregrino Cancer Center, donde se tiene un convenio de participación. De misma manera se tiene una colaboración con la Universidad de Grenoble, donde se ha estado trabajando con datos ómicos.


\subsection{Planteamiento del problema}

La conjunción del aprendizaje profundo en el área biomédica recientemente está dando resultados, como es una tecnología relativamente nueva, existen múltiples problemáticas por abordar como la alta dimensionalidad de datos, datos desequilibrados, explicabilidad de los modelos, estandarización de datos de las bases públicas, la imputación de datos y la clasificación errónea. A continuación se muestran los puntos relevantes que se consideran desafíos:

Los modelos de aprendizaje profundo necesitan grandes cantidades de datos para entrenamiento y validación, por lo que se encuentran desafíos en la disponibilidad de datos a pesar de existir bases públicas estos datos no representan a la diversidad de la población, la calidad de los datos no es confiable al presentar ruido, valores faltantes y errores de medición, y el alto volumen y dimensiones de estos necesitan una etapa de preparación previa al utilizarlo en el modelo de aprendizaje profundo.

Los modelos de aprendizaje profundo son considerados modelos de caja negra al no saber como se toman decisiones que llevan al resultado y la complejidad computacional que se puede presentar en la programación, utilización y entrenamiento de los modelos.

Para destinar el aprendizaje profundo en aplicación clínica se debe asegurar la precisión y la utilidad para los médicos especialistas es por esto que se necesita la colaboración interdisciplinaria entre ingenieros y médicos para demostrar su eficacia y seguridad en entornos clínicos reales.

Los puntos que se plantean abordar en este trabajo de tesis son la codificación de datos en un formato que pueda ser interpretado y analizado por el modelo de red neuronal, donde se considera la normalización de los datos adquiridos, imputación y reducción de dimensiones con el fin de aumentar la precisión del modelo de DL.

Los algoritmos de aprendizaje profundo en aplicaciones de clasificación y predicción utilizando datos ómicos están lejos de ser óptimos debido a la complejidad de los tipos de datos y los problemas antes mencionados, lo que se busca abordar es en la propuesta de un nuevo algoritmo que tome las ventajas que tienen otros e incorporarlas, ya que como se propone vincular con el área médica se requiere una alta precisión en las respuestas que se obtienen del modelo.

\subsection{Propuesta de solución}

La propuesta de solución consiste en proponer un algoritmo de aprendizaje profundo que tenga el potencial de utilizarse clínicamente en diagnóstico y pronostico usando datos ómicos en este caso en particular enfocados en cáncer como lo son genómicos, transcriptómicos y proteómicos. Este modelo se basará en los algoritmos CNN y RNN así como también sus variantes, tomar las características más relevantes de los algoritmos reportados, se tendrá en la salida el propósito de clasificación y predicción, utilizando las bases de datos públicas y de ser posible con datos privados. Se realizará la validación por comparación con los algoritmos existentes y también en un entorno clínico con opinión de médicos especialistas.

De misma manera se propone buscar e incorporar técnicas de tratamiento de datos y codificación que sirva para reducir variaciones en los datos con el fin de incrementar la precisión, entrenamiento y validación del modelo.


\subsection{Objetivo general}

\begin{itemize}

   \addtolength{\itemsep}{-4mm} %con esto se ajusta el interlineado entre la lista
        \item Modelar datos genómicos, transcriptómicos y proteómicos utilizando técnicas de aprendizaje profundo, empleando redes neuronales de diseño propio, y configurar la salida del modelo para propósitos de clasificación y predicción, enfocados en el diagnóstico y pronostico del cáncer.


    \end{itemize}


\subsection{Objetivos específicos}

\begin{itemize}

   \addtolength{\itemsep}{-4mm} %con esto se ajusta el interlineado entre la lista
        \item Codificar datos genómicos, transcriptómicos y proteómicos para ser alimentados en las redes neuronales.
        \item Implementar redes neuronales convolucionada (CNN) y recurrente (RNN), y entrenarlas.
        \item Proponer y diseñar una red neuronal propia a partir de las dos anteriores.
        \item Configurar en cada caso la salida, para propósito de clasificación y predicción.
        \item Enfocar el modelado de las redes neuronales para fines biomédicos en diagnóstico y pronostico del cáncer.
        \item Validar los resultados con las bases de datos e incluyendo opinión de especialistas.
    \end{itemize}


%\subsection{Alcances y limitaciones}

%\subsection{Pregunta de investigación}

%¿Es posible proponer un modelo de red neuronal basado en aprendizaje profundo que ayude a aumentar la precisión en la clasificación/predicción de un fenotipo utilizando datos genómicos, transcriptómicos y genómicos que pueda utilizarse en el área biomédica?

\subsection{Justificación}

%El uso de datos ómicos de diversas fuentes públicas presenta problemas como la heterogeneidad y datos desequilibrados (datos faltantes y/o mal etiquetados), con el uso de algoritmos de aprendizaje profundo enfocados en la imputación y codificación se podría mejorar la eficiencia de predicción y clasificación del modelo para pronóstico del cáncer. La integración con el área biomédica supone una ventaja en el área biomédica para una evaluación temprana en pacientes con un tipo de cáncer y determinar la progresión con el cual se pueden establecer tratamientos adecuados.

El propósito es desarrollar un modelo de aprendizaje profundo que pueda servir como herramienta de toma de decisiones en un entorno clínico, de misma manera, contribuir en la investigación de la incorporación de técnicas de ingeniería en medicina de precisión. Por medio de técnicas de codificación y tratamiento en los datos mejorar la capacidad del modelo a predecir y clasificar con precisión, incluyendo la experiencia de especialistas para reducir discrepancias en el diagnóstico y pronostico del cáncer, permitiendo establecer tratamientos adecuados y a tiempo.

\subsection{Aportación}

La aportación es directamente en proponer y diseñar un modelo de aprendizaje profundo usando datos ómicos que sea posible incorporar en el área médica para la toma de decisiones en el pronóstico y diagnóstico del cáncer, contribuyendo en el área de ingeniería biomédica.
\subsection{Metas}

\begin{itemize}

    \addtolength{\itemsep}{-4mm} %con esto se ajusta el interlineado entre la lista
        
         \item Un modelo basado en aprendizaje profundo con capacidad de aplicación en el área médica que sirva como herramienta de apoyo a médicos especialistas y oncólogos, utilizando datos ómicos relacionados con el cáncer.

         \item Publicación de artículo en una revista indizada en el área de inteligencia artificial con relación en aplicaciones médicas.
         
\end{itemize}
         


\subsection{Metodología}

A continuación se describe la metodología que se seguirá en este trabajo:

Primeramente, se está realizando la reproducción de trabajos realizados para conocer el conjunto de datos de entrenamiento que se utiliza, los algoritmos utilizados y validar los resultados reportados.

De las bases de datos existentes ya se tienen ubicadas las que cuentan con libre acceso de descarga y las frecuentemente utilizadas por investigadores, de las cuales se cuentan distintos estudios ómicos y resultados clínicos.

Realizar la limpieza de datos como manejar valores faltantes mediante métodos de imputación, reducir ruido de medición y normalizarlos.

Por medio de la codificación integrar distintos datos ómicos y métodos para la reducción dimensional y con colaboración con especialistas identificar las características relevantes y utilizar métodos estadísticos para hacer la selección.

Para el diseño del modelo se seleccionarán los modelos más adecuados para las tareas de pronóstico y diagnóstico, tomar las características más relevantes y trabajar sobre un modelo propio.

Definir la arquitectura del modelo y complejidad del modelo, selección de la función de perdida y optimizador adecuado para el entrenamiento del modelo.

Realizar pruebas de entrenamiento monitoreando el rendimiento en el conjunto de validación y realizar el ajuste de hiperparámetros.

Evaluar el rendimiento y precisión del modelo con el conjunto de prueba, utilizando métricas ya definidas así como análisis de sensibilidad para verificar la robustez en la variación de datos.

Para la validación clínica se contempla incluir a los especialistas y de ser posible incluir datos independientes para corroborar los resultados obtenidos así como también tener una generalización. Por último hacer una comparación con los modelos de aprendizaje profundo existentes así como también con los métodos de diagnóstico tradicionales de ser posible.










